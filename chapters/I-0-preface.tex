Over the last century, with the advent of quantum physics and special relativity, the field of particle physics has made tremendous progress in the understanding of the elements and the forces that compose our universe. While many questions remain open for future generations to answer, scientists have continuously been improving the models developed to describe the building blocks of nature and their interactions. Benefiting from the technological advancements led by the advent of the transistor and modern computing, physicists have been given the tools required to create more powerful and complex experiments. These advancements allow scientists to confront theoretical predictions and experimental observations while probing deeper the secrets of the universe. \\

In 2013, the Nobel Prize in Physics was awarded to Fran\c{c}ois Englert and Peter W. Higgs ``\textit{for the theoretical discovery of a mechanism that contributes to our understanding of the origin of mass of subatomic particles, and which recently was confirmed through the discovery of the predicted fundamental particle, by the ATLAS and CMS experiments at CERN's Large Hadron Collider}``. For the thousands of physicists who helped make this discovery possible, the news was received as the recognition of a search spanning over 40 years. It started in 1964 when Fran\c{c}ois Englert together with Robert Brout \cite{PhysRevLett.13.321} and, independently, Peter W. Higgs \cite{PhysRevLett.13.508} proposed a mechanism to explain possible mass differences among the gauge bosons of the Standard Model. As a consequence they postulated the existence of a new elementary particle. They thereby solved the ever present question of the origin of mass in the theory as it was formulated, allowing the field of particle physics to leap forward. However, their hypothesis remained unproven until July 4th 2012, day on which two international collaborations of more than 3000 scientists announced that, after two years of experimentation and data analysis, they had discovered the so called Higgs boson \cite{PhysRevLett.114.191803}, the missing particle which would validate the famed theory. \\

The discovery of the Higgs boson not only solved the problem of the origin of mass of elementary particles but also strengthened the foundations of the Standard Model, the prominent theory in particle physics also considered to be one of the greatest achievements in the field. The Standard Model is a mathematical construction that classifies the elementary particles and the interactions they undergo. It was developed in the 1970s driven by the idea of grand unification: the description of all forces of nature by a single theory. It combined the electroweak and quantum chromodynamics theories, respectively describing the electromagnetic and weak, and strong interactions. The Standard Model quickly gained credibility with the discoveries of neutral weak currents by the Gargamelle experiment and the charm quark by the Stanford Linear Accelerator Center and the Brookhaven National Laboratory in 1974. Over the years, many other theoretical predictions have proven to be correct through experimental results and discoveries. These successes were made possible thanks to the construction of new experiments that allow physicists to reach new frontiers in the search for answers. \\

This introduction to particle physics aims to provide the reader with a background on both the theoretical and experimental aspects of the field. Chapter \ref{chap:I-1-standard-model} covers the Standard Model and the mathematical tools used to describe the physics supporting it. After a review of the current state of the theory and the description of nature it yields, the limitations of the model are addressed along with the solutions that are currently being investigated. Focus is then given to the experimental side of particle physics in Chapters \ref{chap:I-2-lhc} and \ref{chap:I-3-cms} which respectively describe the Large Hadron Collider at CERN, the European Organization for Nuclear Research, and the Compact Muon Solenoid experiment. The Large Hadron Collider is state-of-the-art in particle collider engineering which, through the collisions it provides, enables the Compact Muon Solenoid detector to probe the mysteries of nature.
