\chapter{A Classical Approach to Data Acquisition System Design}
\label{chap:III-1-arch}

  In the early development stages of the DAQ system for the GE1/1 detector, before the first version of the OptoHybrid was designed, a small prototyping setup was developed to readout a 10 cm $ \times $ 10 cm Triple-GEM detector using VFAT2 Hybrids. With time, the setup was improved to use the GLIB and later on the OptoHybrid. Next to the data readout chain, the system also controls the high voltage and gas sources from a web interface. \\

  In this chapter, we describe the evolution of the DAQ system developed to read out a small Triple-GEM prototype. We describe the technologies that have been used and the developments that were performed in order to integrate the components in the system.

  \section{The Experimental Setup}

  \section{The Infrastructure of the Data Acquisition System}

    \subsection{The Central Computing Node}

    \subsection{The Web Interface}

    \subsection{The High Voltage and Gas Controller}

  \section{A First Prototype using the Xilinx SP601 Development Board}

  \section{Upgrade of the System using the GLIB}

  \section{Final System using the First Prototype of the OptoHybrid}
