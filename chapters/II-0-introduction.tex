%% Numbers    checked
%% Spelling   checked
%% Acronyms   checked

In the upcoming years, several upgrades of the LHC and its injection chain are planned with the aim of increasing the performance of the machine. The next upgrade, LS2, is currently set to take place in 2019 and will increase the instantaneous luminosity of the LHC to 2 $ \times $ 10$^{34}$ cm$^{-2}$ s$^{-1}$ for the 2020-2022 run. In 2023, the LHC Phase 1 will end with a total integrated luminosity of around 300 fb$^{-1}$ and LS3 will start, allowing to prepare the systems for the so called high-luminosity LHC or LHC Phase 2. The LHC will restart mid-2025 with an instantaneous luminosity of 5 $ \times $ 10$^{34}$ cm$^{-2}$ s$^{-1}$. \\

The muon spectrometer of CMS must be able to cope with the program of the LHC. It was originally designed to be a hermetic and redundant system relying on the DTs, CSCs, and RPCs to provide efficient muon detection, identification, and triggering. The DTs are installed in the barrel and cover a region of |$\eta$| < 1.2, and the CSCs are located in the endcaps between 1.0 < |$\eta$| < 2.4. Additionally, the RPCs provide redundancy in both the barrel and the endcaps but were not instrumented in the |$\eta$| > 1.6 region due to concerns about their tolerance to high fluxes of particle. \\

Besides the lack of redundancy, studies have shown that the triggering efficiency of the muon system of CMS will be drastically affected by the increase in luminosity and reduce the physics performance by limiting the parameter phase-space that can be studied. \\

To tackle these challenges, the CMS GEM collaboration \cite{Colaleo:2021453} will install during LS2 a set of muon detectors that use the Gas Electron Multiplier (GEM) technology in the 1.6 < |$\eta$| < 2.2 region left vacant by the RPCs. The objective is to improve the L1 trigger efficiency in combination with the CSCs. To this end, a new DAQ system has to be designed based on the micro telecommunications computing architecture standard. Small scale systems have been studied during test beam campaigns and four GEM chambers equipped with the full DAQ system will be installed in CMS during the YETS-2016 to perform a so-called slice test. \\

This part focuses on the CMS GEM upgrade project and the developments of the author with respect to the DAQ system design and characterization. Chapter \ref{chap:II-1-gem} provides an overview of the GEM technology and the status of the project. It brushes the evolution of the chamber design and the different generations of GEM detectors that have been developed, and covers the performances of the detectors. The architecture of the DAQ system is explained in details in Chapter \ref{chap:II-2-daq} describing each component of the system and their integration in the global CMS DAQ system. Chapter \ref{chap:II-3-test-beam} focuses on the test beam campaigns that took place in November 2014 and 2015 and helped to qualify the system providing results on the chamber performance. Following the test beams, modifications have been done to the electronics in order to prepare for the slice test and the final system. Chapter \ref{chap:II-4-slice-test} reviews the architecture that will be installed during the YETS-2016 and the developments done by the author towards the migration to the final electronics. Finally, Chapter \ref{chap:II-5-qualification} and \ref{chap:II-6-irradiation} detail the work performed on the qualification and calibration, and the irradiation tests of the on-detector electronics used in the GEM project. \\

All firmware and software developments, analysis, and results exposed in Chapters \ref{chap:II-3-test-beam}, \ref{chap:II-4-slice-test}, \ref{chap:II-5-qualification}, and \ref{chap:II-6-irradiation}, have been performed solely by the author unless otherwise specified.
