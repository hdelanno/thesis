\chapter{Irradiation Tests}
\label{chap:II-6-irradiation}

  The OptoHybrid will be located in a region of CMS exposed to high fluxes of particles, some of which might interact with the FPGA and generate errors in the logic. Those could in turn influence the functioning of the system and degrade its performance. To solve this issue, irradiation tests have been performed to measure the interaction cross-section of the particles with the various components of the FPGA. Two OptoHybrids have been placed in a high intensity proton beam with a dedicated firmware designed to detect errors. The two boards were each controlled by an additional OptoHybrid placed outside the test area which recorded the events and statistics. \\

  In this chapter, we provide the user with an overview of the internal architecture of an FPGA to better understand the potential sources of errors and the effects of radiation. We then describe the firmware of the irradiated and control FPGAs which has been used during the test. The setup and beam parameters of the irradiation test are reviewed before presenting the results that were obtained after analysis.

  \section{Architecture of an FPGA}

    To optimize the occupancy of the resources of the FPGA and develop code that uses the full potential of the device, a deep comprehension of the intrinsic architecture of the chip is required. Although families of FPGA differ in size and complexity, their building blocks remain the same and are described hereafter.

    \subsection{Configurable Logic Blocks}

      (CLBs) \cite{VIRTEX-CLB} are the base elements of the FPGA used to implement sequential and combinatorial logic. They are composed of two important objects: look-up tables (LUTs) and registers. LUTs are components which outputs are a function of the inputs as defined in a programmable table. They implement a truth table for every possible combination of the inputs which defines the value of the outputs. The response of a LUT to a change in the inputs is almost instantaneous. LUTs in the Xilinx Virtex-6 FPGAs can either implement functions with six inputs and a single output or functions with five inputs and two outputs. The outputs of the LUTs can, if so required by the design, be connect to registers which sample the signals at the rising edge of a given clock. Register are used for their sample-and-hold functionality which makes designs synchronous to clocks. With these two components, CLBs can implement complex functions and describe intricate systems. \\

      \begin{figure}[p!]
        \centering
        \includegraphics[width=\textwidth]{img/II-6-irradiation/clb.png}
        \caption{Simplified view of a slice composed of four LUTs on the left and eight registers on the right \cite{VIRTEX-CLB}.}
        \label{fig:II-6-clb}
      \end{figure}

      Each CLB is composed of two slices each made of four LUTs and eight register which layout is shown in Figure \ref{fig:II-6-clb}. Each LUT is connected to an input bus of six signals (A, B, C, and D inputs) and to an unbuffered output bus of a single bit (O6 to A, B, C, and D). Four additional signals enter the slice (AX, BX, CX, and DX) and are connected to multiplexers (red) which allows to buffer either the former or the second output of the LUTs (O5). The output of the same buffer is then connected to a second multiplexer (orange) which allows to select either the former or an unbuffered output of the LUTs (AMUX, BMUX, CMUX, and DMUX). Finally, a third multiplexer (green) connects a range of signals to the registers on the right which are then connected to four outputs (AQ, BQ, CQ, and DQ). Three additional multiplexers (blue) offer the possibility to mix the signals from the four LUTs to generate a wider range of logical operations. The flexibility offered by this architecture is what enables FPGAs to implement complex designs. The Xilinx Virtex-6 FPGA used in the OptoHybrid v2a (XC6VLX130T) holds 10 000 CLBs with a total of 80 000 LUTs and 160 000 registers.

    \subsection{The switching matrix}

      The inputs and outputs of the slices of the CLBs as well as of the other components of the FPGA are interconnected through the switching matrix. It is a vast network of wires and switches which are used to route signals between elements. The open or closed state of each switch is programmable and defines the routes signals follow inside the logic. Figure \ref{fig:II-6-switch} shows an element of the matrix connected to two slices (blue) along with the signals that are routed in the network (cyan). Complex designs can span a large area of the FPGA due to the high usage of logic and thus be difficult to place and route by the compiler. A technique called floorplanning is used to reduce compilation time and improve the design by constraining parts of the code to given areas in the FPGA.

      \begin{figure}[h!]
        \centering
        \includegraphics[width=0.8\textwidth]{img/II-6-irradiation/switch.png}
        \caption{Schematic of two slices (blue) connected to an element of the switching matrix which routes the signals (cyan) between components over the multitude of existing paths (gray).}
        \label{fig:II-6-switch}
      \end{figure}

    \subsection{Block RAM}

      Besides programmable logic, the FPGA also includes dedicated storage elements. Block RAMs \cite{VIRTEX-RAM} (BRAMs) can store up to 36 kb of data which can be configured in different ways: 32K x 1 bit, 16K x 2 bits, etc. They can also be used as First In First Out (FIFO) modules which are similar to data queues.

    \subsection{Digital Signal Processing}

      Digital Signal Processing units (DSPs) \cite{VIRTEX-DSP} are modules which perform mathematical operations using dedicated hardware elements. They are used to quickly solve problems without relying on CLBs which can consume large amount of resources to perform an equivalent task. Figure \ref{fig:II-6-dsp} shows a diagram of the DSP48E1 slices present in the Xilinx Virtex-6 family. They include a 30-bits adder, a 25-bits by 18-bits multiplier, and a programmable module which can implement either a multiplication, an addition, or a logic operation. Using the various multiplexers and configuration bits, the user can define the data paths that are followed and thus create DSP modules which meets the requirements of the design.

      \begin{figure}[h!]
        \centering
        \includegraphics[width=\textwidth]{img/II-6-irradiation/dsp.png}
        \caption{Diagram of the DSP48E1 slices present in the Xilinx Virtex-6 family \cite{VIRTEX-DSP}.}
        \label{fig:II-6-dsp}
      \end{figure}

    \subsection{The configuration memory}

      The configuration memory holds the configuration of the entire FPGA and defines its behavior. It sets the truth table of the LUTs, parametrizes the DSP, creates the connection between elements, etc. It is what implements the design in the FPGA. In most FPGAs, the configuration memory is volatile and will lose its configuration upon power down or reset. To reload it, the FPGA tries to read it out of an attached memory device or remains in a blank or corrupted state if it fails to do so.

  \section{Effects of Radiation on FPGAs and Mitigation Techniques}

    When charged particles pass through an FPGA, they interact with the silicon substrate and deposit charge within the device. If the interaction takes place near a transistor, as depicted in Figure \ref{fig:II-6-transistor}, the charges can affect the functioning of the component and induce a change of state. Alternatively, particles can produce current or voltage spikes which propagate in the design. Each type of interaction results in different effects in the FPGA. The most common are the Single Event Upsets (SEU) and Single Event Transients (SET). Besides being influenced by punctual events, the FPGA will also age and degrade due to the Total Ionizing Dose (TID) of radiation.

    \begin{figure}[h!]
      \centering
      \includegraphics[width=\textwidth]{img/II-6-irradiation/transistor.png}
      \caption{Charge deposition by a charged particle inside a transistor affecting the state of the circuit \cite{XILINX-RADIATION}.}
      \label{fig:II-6-transistor}
    \end{figure}

    \subsection{Single Event Upset}

      When the charge deposited by radiation is located within a transistor, the critical voltage of the latter can be modified and a change of state occur. In case of memory cells, a change of state corresponds to a bit flip, which can influence the design. If the configuration memory is hit, the description of the CLBs, DSPs, and switch matrix will change dynamically during run time and the implemented logic will thus be affected. For BRAMs, the bit flip will occur in the data stored in memory, resulting in corrupted information. \\

      A common technique used to mitigate SEUs in CLBs and DSPs is to triplicate the logic and couple its outputs to a majority voter. First, the sequential logic is triplicated, meaning the inputs are sent to three identical modules which output is returned synchronously. The latter are then forwarded to a majority voter which performs a bit-by-bit vote and returns the most probable response. This technique can recover data if only one of the modules is affected by an SEU. In case two modules are corrupted and flip the same bit, data will be corrupted as well. Additionally, a flag can be raised by the majority voter if all three inputs are not identical, meaning an error occurred. \\

      To correct SEUs taking place in the configuration memory, the latter has to be read out by the FPGA itself. This is possible through the Soft Error Mitigation (SEM) core, a component placed inside the device which allows the firmware to analyze the data in memory and, using a two bit complement, perform a FEC on the bits. Although very effective to correct errors on the fly, this technique is time consuming due to the large amount of data to readout and can take up to several milliseconds to identify and repair and error. Furthermore, the FEC will fail to recover the data in case two bits are flipped within the same word, raising a flag that tells the system a hard reset is needed. \\

      Finally, SEUs occurring in BRAMs can be corrected using a two complements FEC which can recover single bit errors and detect double bit errors.

    \subsection{Single Event Transients}

      If the event affects combinatorial logic, a spike in current or voltage can be produced and propagate within the design. When coupled to sequential logic, such errors can be easily recovered from if their duration is small in comparison to the frequency of the clock used to run the circuit. Unless the error is produced exactly at the sampling time of the registers, it is not stored and thus mitigated.

    \subsection{Total Ionizing Dose}

      With time, the FPGA ages under the effects of radiation and malfunctions might appear. As charges get trapped in the substrate, transistors inside the device suffer from a shift in the threshold voltage of the gate and of leakage currents. This in turn affects the design, either by rendering data cells useless or by causing changes in the gate states.

  \section{Firmware Design for the Irradiated FPGA}

    The firmware of the irradiated FPGA is designed to use a maximum of the available resources and transmit any error detected during run time. Specific firmware has been developed to test the CLBs, BRAMs, DSPs, and SEM. To optimize the design, floorplanning is used to divide the FPGA in ten regions running identical code. This prevents the compiler from placing elements in different sectors of the device and thus increase routing resources. Figure \ref{fig:II-6-floorplanning} depicts how the design occupies the FPGA: the image on the left represents the FPGA sectors labeled XaYb composed of CLBs in dark blue, BRAMs in red, and DSPs in green; the image in the middle highlights the resources used by the design; and the image on the right shows how the firmware developed for each function occupies the FPGA with CLBs in blue, BRAMs in yellow, DSPs in red, SEM in purple, and the communication protocol in green.

    \begin{sidewaysfigure}[p!]
      \centering
      \includegraphics[width=0.32\textwidth]{img/II-6-irradiation/fpga-empty.png}
      \includegraphics[width=0.32\textwidth]{img/II-6-irradiation/fpga-used.png}
      \includegraphics[width=0.32\textwidth]{img/II-6-irradiation/fpga-color.png}
      \caption{Schematic view of the occupancy of the FPGA. Left: sectors of the FPGA labeled XaYb composed of CLBs in dark blue, BRAMs in red, and DSPs in green. Middle: highlight of the resources used by the design. Right: occupancy of the code developed to test the CLBs in blue, BRAMs in yellow, DSPs in red, SEM in purple, and the communication protocol in green.}
      \label{fig:II-6-floorplanning}
    \end{sidewaysfigure}

    \subsection{Configurable Logic Block}



    \subsection{Block RAM}

    \subsection{Digital Signal Processing}

    \subsection{Soft Error Mitigation}

  \section{Firmware Design for the Control FPGA}

    \subsection{Communication Protocol}

    \subsection{ChipScope Core}

  \section{Irradiation Setup}

  \section{Data Analysis}

  \section{Conclusion}
