The backbone of every DAQ system is the integration of its components into a single coherent architecture. The latter must provide a seamless user experience abstracting every aspect of the design and be optimized for efficient data readout and transfer. While companies provide all-in-one system that are well integrated and include intuitive user interfaces, they are rarely suitable for custom applications that need to interface with components from different vendors. Custom interfaces need to be developed to control the system and transfer data between the subsystems. \\

The architecture of a DAQ system is mainly defined by the way data is distributed between the components: it is either pulled or pushed. Data is pulled from one component to another when a transfer solely occurs upon request from the receiver, and is pushed when the decision to forward data is taken by the transmitter. As each component either pushes or pulls data, a bottleneck appears in the system due to limitations in the transfer bandwidth, storage capacity, etc. This in turn can cause data losses or system failures if not handle properly. Classical approaches to DAQ system design make use of a central node which handles the system and coordinates the communication between the components. However, new technologies have enabled the development of new topologies which blur the lines with components which fulfill multiple roles. \\

In this part, two DAQ system are presented: one following the classical approached which uses a central node, and the other using recent developments in web technologies to create a modern and dynamic system.
