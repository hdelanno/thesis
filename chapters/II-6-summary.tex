\chapter{Summary}
\label{chap:II-7-summary}

  The Triple-GEM upgrade project aims at improving the performance of the muon spectrometer of CMS which will suffer from the increase in luminosity of the LHC in the coming years. The installation of the GE1/1 stations has been approved by the collaboration and will occur during LS2. To prepare for the integration in CMS, a small scale test will take place during the YETS 2016 involving the instrumentation of four superchambers in one endcap of the detector. \\

  In the CMS GEM collaboration, the GEM DAQ group focused on the design of the DAQ system of the detectors. Within this group, we were in charge of the firmware design of the on-detector and off-detector electronics, namely the OptoHybrid and GLIB boards. We developed both a flexible firmware architecture which provides the user with numerous monitoring and control options, and a web application which allows the user to control the system dynamically. After a successful integration of the components within a stable system, they were tested during two test beam campaigns in November 2014 and November 2015. These demonstrated that the designed DAQ system is able to handle a superchamber of GE1/1 detectors with ease. Furthermore, they enabled the recording of data that yielded various measurements of the efficiency and other significant parameters of the detector. During the analysis of the data, we performed various studies of the noise and efficiency levels against high-voltage, threshold, and particle rates. We obtained a single chamber efficiency of 97\% and tested the rate capability up until a trigger rate of 120 kHz with a resulting efficiency of 96\%. The analysis we have performed show that the GEM detectors equipped with the DAQ system we have designed meet the requirements of the project regarding the efficiency and the rate capability of the chambers. \\





For the first time, a method was developed to use the VFAT2 capabilities to their full extend by taking advantage of the channel by channel optimization. The procedure is used to detect faulty units and perform qualification tests which results are stored in database for future reference. With this script, the analog front-end of each VFAT2 is entirely characterized which provides information on the effective threshold applied in terms of electrons and thus induced charge. \\

Furthermore, a GEB testing board has been design to test each position on the GEB and detect broken lines. The board relies on an FPGA coupled with an MCU to communicate with the OptoHybrid and test the integrity of the transmitted signals. The results of each test are displayed using on-board LEDs. \\

Finally, the system as a whole is tested using a script which targets specific components of the architectures. Random read/write operations to the GLIB, OptoHybrid, and VFAT2s are performed, tracking data is read out, and stress tests are done to push the system to the limit. \\

These tools are used for the preparation of the slice test to select appropriate components and install testing facilities at CERN and in other associated research laboratories.




To characterize the behavior of the on-detector electronics used for the CMS GEM project when subject to radiation, two OptoHybrid v2a boards were exposed to proton beams during irradiation tests performed at the cyclotron of Louvain-La-Neuve. Custom firmware was developed for the FPGAs in order to compute the interaction cross section with the various components inside the chip and to study the effectiveness of mitigation techniques.  \\

After analysis of the data, we obtained an interaction cross section of 3.08 $ \times $ 10$^{-7}$ cm$^{2}$ and 1.02 $ \times $ 10$^{-7}$ cm$^{2}$ for the configuration memory of the FPGA and the BRAM respectively, the two main sources of errors. When transposed to the environment of CMS, this yields a rate of 27 errors a day and 9 errors a day respectively. These errors can be mitigated using triplication techniques which have proven to be efficient at low particle fluxes. \\

Over the course of the tests, the FPGAs were exposed to a total of 84 kRad, corresponding to a dose 8 times higher than what will be collected during the totality of the LHC Phase 2. We did not observe any increase in the interaction cross section for the components of the FPGAs which continued to function correctly until the end of the tests. \\

From our studies, we conclude that, although subject to SEUs, the FPGAs used in the OptoHybrid design are suitable for the environment of CMS and will survive for the entirety of the Phase 2 run. The mitigation technique that we tested, namely triplication, is a suitable method to mitigate any error in the data accompanied by the use of the SEM core to correct the upsets that modify the configuration of the FPGA.
