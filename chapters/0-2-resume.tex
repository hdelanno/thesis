\chapter*{Résumé}
\addcontentsline{toc}{chapter}{Résumé}

  Cette thèse de doctorat s'inscrit dans le projet de mise à niveau du spectromètre à muons du Compact Muon Solénoid (CMS) auprès du Large Hadron Collider (LHC). Après un arrêt pour maintenance prévu en 2018-2019, le LHC reprendra son programme de recherche à une luminosité de 2 $\times$ 10$^{34}$ cm$^{-2}$ s$^{-1}$, soit deux fois sa valeur nominale. Ceci aura pour conséquence d'accroître le taux de particules auquel seront soumis les détecteurs de CMS et d'entraver l'efficacité de détection de ces derniers. Le spectromètre à muons de CMS sera tout particulièrement touché à cause du manque de redondance dans sa partie avant. Afin de palier à ce problème, il est proposé d'installer des détecteurs Gas Electron Multipliers (GEM) dans la première station à muons. La technologie GEM répond aux besoins de CMS en offrant une excellente efficacité de détection à de hauts flux de particules. Au sein de la collaboration GEM, le sous-groupe en charge du système d'acquisition de données (DAQ) doit développer l'électronique et les logiciels de gestion de la chaine de lecture des détecteurs. \\

  Nous présentons ici le travail de l'auteur réalisé en tant que principal développeur firmware pour l'électronique du système DAQ. Ces développements visent à créer une architecture qui achemine les données depuis l'électronique de lecture du détecteur jusqu'aux systèmes situés dans la zone de comptage, tout en offrant la possibilité de contrôler et surveiller l'ensemble des composants du DAQ. Le système mis en place a été largement testé durant deux campagnes de tests en faisceaux qui ont fourni des informations concernant la stabilité du DAQ ainsi que des mesures de l'efficacité des détecteurs. Nous décrivons également l'ensemble des travaux réalisés afin de caractériser les composants électroniques avant leur installation dans CMS ainsi que les résultats des tests d'irradiation effectués sur l'électronique du détecteur. Ces derniers permettent de mieux comprendre les conséquences des radiations sur le système et de développer des méthodes de mitigation. \\

  De plus, le travail de l'auteur sur la création d'une nouvelle architecture de système DAQ est décrit. Cette dernière combine les avancées récentes en terme de technologies web à une topologie de réseaux non-classique afin d'accroitre la bande passante disponible et de créer un système en temps réel.
