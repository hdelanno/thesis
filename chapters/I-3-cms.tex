\chapter{The Compact Muon Solenoid Experiment}
\label{chap:I-3-cms}

	The Compact Muon Solenoid (CMS) \cite{1748-0221-3-08-S08004} is a multi-purpose particle detector recording the collisions provided by the LHC. It was, along with ATLAS, the first experiment officially approved for the LHC by the CERN research board in January 1997 after a long evalution process of the letter of intent published four years earlier by the two collaboration. The construction of the detector started in 2005, after the excavation works of the cavern finished, and spanned until 2008. Since then, CMS has been proficiently taking and analyzing data, and announced on July 4th 2012 the discovery of the Brout-Englert-Higgs boson, one of the most significant results of the collaboraton.

  \section{Overview}

    At nominal energy and luminosity, the LHC produces around 20 proton-proton collisions per BX which results in around 1000 particles in the final state. In order to distingish physical processes with great precission CMS has to ensure good identification and reconstruction of the particles. To this end, the detector was built around four requirements:
    \begin{itemize}
      \item Good muon identification and momentum resolution over a wide range of momenta and angles, good dimuon mass resolution ($ \approx $ 1\% at 100 GeV), and the ability to determine unambiguously the charge of muons with p < 1 TeV;
      \item Good charged-particle momentum resolution and reconstruction efficiency in the inner tracker. Efficient triggering and offline tagging of $ \tau $'s and b-jets, requiring pixel detectors close to the interaction region;
      \item Good electromagnetic energy resolution, good diphoton and dielectron mass resolution ($ \approx $ 1\% at 100 GeV), wide geometric coverage, $ \pi^0 $ rejection, and efficient photon and lepton isolation at high luminosities;
      \item Good missing-transverse-energy and dijet-mass resolution, requiring hadron calorimeters with a large hermetic geometric coverage and with fine lateral segmentation. \\
    \end{itemize}

    \begin{figure}[h!]
      \centering
      \includegraphics[width=\textwidth]{img/I-3-cms/cms.png}
      \caption{Schematic representation of the Compact Muon Solenoid detector installed at LHC \cite{1748-0221-3-08-S08004}.}
      \label{fig:I-3-cms-global-view}
    \end{figure}

    These requirements are met through the subdivision of CMS in various detection systems each specialised in the reconstruction of a given type of particles. The overall layout of CMS, shown in Figure \ref{fig:I-3-cms-global-view}, is divided into the barrel and the two endcaps, regions where the detectors are respectivly placed in parrallel and perpendicularly to the beam pipe. At the center of the detector, closest to the interaction point, lies the inner tracking system. Composed of 3 layers of silicon pixels and 10 layers of silicon strips detectors, it is designed to detect the passage of any charged particle with high precision. Surrounding the tracking system are the electromagnetic and hadronic calorimeters which respectivly measure the energy of electrons and photons, and hadrons. These detectors are placed inside a superconducting solenoid magnet which produces a strong 3.8 T field that bends charged particles and allows for precise momentum measurments. Outside of the magnet, three different technologies of muon detectors are placed on large iron yokes. Furthest from the interaction point are the very-forward calorimeters which intercept particles with low flighing angle.

    The coordinate system used in CMS has its origin at the nominal interaction point of the beams, the y-axis pointing upwards, the x-axis pointing toward the center of the LHC, and the z-axis directed along the beam direction. The polar coordinates (r, $ \phi $) are defined in the x-y transverse plane and the widly used pseudorapidity $ \eta $ is taken to be
    \begin{equation}
      \eta = - \ln\left( \tan\left( \frac{\theta}{2} \right) \right) ,
    \end{equation}
    where $ \theta $ is the azimuthal angle between the z-axis and the transverse plane.

  \section{The Inner Tracking System}



  \newcommand{\GeVc}{GeV c$ ^{-1} $}
  \newcommand{\um}{$ \mu $m}
  \newcommand{\us}{$ \mu $s}
  \newcommand{\pT}{$ p_T $}
  \newcommand{\pZ}{$ p_Z $}
  \newcommand{\axis}[1]{#1}

  The silicon tracker is the detector closest to the IP. It is composed of two different types of semiconductor detectors: \emph{silicon pixels} and \emph{silicon strips}. These detectors have an excellent spatial resolution (down to 25 \um{}) which yields excellent momentum reconstruction capabilities (resolution of the order of 1\% at low transverse momenta). The disposition of the different technologies is represented in Figure \ref{fig:lhc_and_cms__cms_tracker}. The silicon pixels are represented in blue, while TIB, TID, TOB, and TEC refer to different regions of the silicon strip detector. \\

  \begin{figure}[h!]
    \centering
    % \includegraphics[width = 13cm]{2_LHC_CMS/img/cms_tracker_view.png}
    \caption{Disposition of the different detectors in the silicon tracker. PIXEL (blue) refers to silicon pixel detectors while TIB, TID, TOB and TEC (red) all refer to silicon strip detectors \Cite{CMS_at_LHC}.}
    \label{fig:lhc_and_cms__cms_tracker}
  \end{figure}

  Semiconductor detectors are made out of two pieces of silicon, one negatively doped containing more unbounded electrons, and one positively doped containing more unbounded holes (absence of electrons), put together to form a n-p junction. At the interface, the electrons and the holes diffuse in the opposite region and recombine with the particles of opposite charge, creating an unbalance in charge: the n region and p region close to the junction become, respectively, positively and negatively charged. From this, an electric field is formed which slows the diffusion down until the system reaches equilibrium. When a charged particle passes through this region and losses energy, electrons switch from non-conductive to conductive bands creating electrons/holes pairs. Under the action of the electric field, they migrate towards the n or p regions and form the signal on the readout electronics. Unfortunately, the number of unbounded charges is still high compared to the formed signal and the active region is small. To increase the detection efficiency, a voltage difference is applied to the semiconductor further diffusing the electrons and holes through the junction. Typically, for a voltage difference of 100 V, the size of the active region is of the order of 300 \um{}.

  \subsubsection{Silicon Pixel Detectors}
  \label{sec:lhc_and_cms__silicon_pixel_detectors}

    The silicon pixels detectors, represented in Figure \ref{fig:lhc_and_cms__cms_pixel_detector}, are composed of small 100 \um{} x 150 \um{} rectangles of readout material disposed on a block of detection medium formed by a n-p region. The electrons are formed in that region and migrate towards the silicon pixels. The challenge arising is that each pixel needs its own readout electronics which takes a significant amount of space and requires output cables. These cables prevent the placing of detectors which creates dead-zones. Physicists and engineers must find the right balance between the number of pixels (granularity), the size of the electronic, and the detectors' resolution. \\

    \begin{figure}[h!]
      \centering
      % \includegraphics[width = 8cm]{2_LHC_CMS/img/cms_tracker_pixel.png}
      \caption{Disposition of the readout pixels (orange and red) on the same detection block (gray) inside a silicon pixel detector \Cite{CMS_Tracker_Construction}.}
      \label{fig:lhc_and_cms__cms_pixel_detector}
    \end{figure}

    Nevertheless, the pixel detectors are the most precise tracking technology in CMS with a spatial resolution of 15 to 20 \um{}. This value is smaller than the size of the pixels because of \emph{charge sharing}. All the pixels sharing the same detection block also share the energy deposited by a particle. By reading the total deposited charge on each of the pixels, we can find the \emph{Center of Gravity} (COG). The COG method is based on the assumption that there exists a linear relation between the induced pulse's height on a pixel and the distance between its center and the particle's hit, so that each pixel is assigned a weight proportional to the deposited charge. The reconstructed coordinate $ \mathbf{x}_{COG} $ of the cluster is then given by
    \begin{equation}
      \mathbf{x}_{COG} = \frac{\sum_i \mathbf{x}_i q_i}{\sum_i q_i} \ ,
      \label{eq:lhc_and_cms__charge_sharing}
    \end{equation}
    where $ q_i $ is the individual pixel signal in the cluster, and $ \mathbf{x}_i $ is the positions of the pixel in the defined coordinate system. The same technique can be applied to other detectors as long as the charge is read out analogically and not digitally\footnote{One can consider using the same method with digital readouts but will not be able to achieve the same resolutions.}.

  \subsubsection{Silicon Strip Detectors}
  \label{sec:lhc_and_cms__silicon_strip_detectors}

    The most outer layers of the tracker cannot have a granularity as high as the pixel detectors, for both financial and technical reasons. The amount of data that would have to be read out is considerable, and the technology to do so is not yet available. A way to reduce the granularity of the detectors is to measure only one coordinate by using silicon strips instead of pixels. The strips are separated by 80 to 122 \um{} which gives a resolution between 23 \um{} and 53 \um{} in the direction perpendicular to the strips. Unfortunately, as expected, there is a larger error on the other coordinate (along the strip) corresponding to the size of the detection cell. To improve global precision, some of the cells have two strip detectors placed with a small stereo angle, typically 100 mrad, allowing them to measure both coordinates. However, this set up generates \emph{ghosts} because of the ambiguity created when more than one particle hits the detector at the same time.

  \subsubsection{System Performances}
  \label{sec:lhc_and_cms__tracker_system_performances}

    Due to the magnetic field generated by the solenoid, the trajectories charged particles are bent inside the tracker. The relation between the bending radius of the track $ R $ , the transverse momentum $ p_T $, and the intensity of the magnetic field $ B $ is
    \begin{equation}
      R[\mbox{m}] = \frac{p_T[\mbox{GeV c}^{-1}]}{0.3 B[\mbox{T}]} \ .
      \label{eq:lhc_and_cms__radius_to_momentum_relation}
    \end{equation}
    By measuring the bending radius of the track and inverting the previous relation, the transverse momentum of the particles can be obtained. Tracks created by high energy particles will be straighter than those left by low energy particles and therefore more difficult to reconstruct. \\

    As previously stated, the tracker offers an excellent resolution on the position of the particles and therefore gives precise measurements of the particles' momentum. Figure \ref{fig:lhc_and_cms__cms_tracker_performances} shows the resolution on the transverse momentum \pT{} (left) and detection efficiency (right) of the tracker as a function of the pseudo-rapidity $ \eta $ for muons of transverse momenta \pT{} of 1, 10, and 100 \GeVc{}. The resolution is less than 1\% for muons of 1 and 10 \GeVc{} in the barrel ($ \eta $ < 1) and quickly rises in the most forward region. The same effect is observed for the detection efficiency which is close to 100\% in the barrel but significantly diminishes at higher pseudo-rapidities. Precision decreases in the most forward region where the strip pitch is greater and more material is present, causing more scatterings.

    \begin{figure}[h!]
      \centering
      % \includegraphics[width = 6.3cm]{2_LHC_CMS/img/cms_tracker_resolution.png}
      % \includegraphics[width = 6.3cm]{2_LHC_CMS/img/cms_tracker_efficiency.png}
      \caption{Resolution on the transverse momentum \pT{} (left) and detection efficiency (right) of the tracker as a function of the pseudo-rapidity $ \eta $ for muons of transverse momenta \pT{} of 1, 10, and 100 \GeVc{} \Cite{CMS_at_LHC}.}
      \label{fig:lhc_and_cms__cms_tracker_performances}
    \end{figure}



  \section{The Electromagnetic Calorimeter}

  Calorimeters are devices that absorb the full kinetic energy of a particle by creating particle cascades, and provide a signal that is proportional to that deposited energy. \\

  Two types of calorimeters are present in CMS: the electromagnetic to detect electrons and photons, and the hadronic to detect hadrons. Each of them corresponding to two of the elementary interactions through which particles can interact with the medium: electromagnetic interaction and strong interaction. \\

  The parameters characterizing particles showers are the radiation length $ \lambda_R $, and the Molière radius $ R_M $ for the electromagnetic calorimeter, and the absorption length $ \lambda_a $ and the interaction length $ \lambda_I $ for the hadronic calorimeter, all depending upon the atomic properties of the material.
  \begin{enumerate}
    \item[] $ \lambda_R $ is the distance that an electron or photon has to travel inside the calorimeter to, respectively, emit a photon or create an electron/positron pair.
    \item[] $ R_M $ gives the radius of the cylinder in which 90\% of the electromagnetic shower is contained.
    \item[] $ \lambda_a $ is the average distance that a hadron has to travel before undergoing an inelastic interaction with the medium.
    \item[] $ \lambda_I $ yields the distance after which a hadron will have scattered inelastically and also gives the radius of the cylinder in which 95\% of the hadronic shower is contained.
  \end{enumerate}
  Those parameters result in the spatial extension of the cascades giving an idea of the granularity needed to correctly distinguish showers. Because the interaction length $ \lambda_I $ of hadrons is much larger than the radiation length $ \lambda_R $ of electrons and photons, the ECAL is placed first. This also implies that hadrons can pass through the ECAL without interacting, or at least not significantly, even if it is multiple $ \lambda_R $ long. \\

  The energy resolution of calorimeters depends upon the number of particles in the cascade hence the energy of the particle
  \begin{equation}
    \left( \frac{\sigma_E}{E} \right)^2 = \left( \frac{a}{\sqrt{E}} \right)^2 + \left( b \right)^2 + \left( \frac{c}{E} \right)^2 \ ,
  \end{equation}
  where $ a $ is the stochastic term depending upon the development of the shower and the detector's response, $ b $ is the constant term determined by the calibration and the uniformity of the crystal, and $ c $ is the noise term from the electronics. Unlike the tracker, the calorimeters' resolution increases with the energy, offering the best resolution at high energies.

  The two main processes allowing the detection of electrons and photons are respectively Brëmsstrahlung and pair creation. These occur as long as the resulting particles (electrons and photons) have enough energy to repeat the process, creating an electromagnetic cascade inside the material. The size of the cascade hence the number of photons emitted by the scintillator is proportional to the energy of the incident particle. Muons do not significantly interact with the ECAL because the radiative processes are greatly suppressed. Indeed, Brëmsstrahlung is proportional to m$ ^{-2} $ (inverse-squared mass) and is therefore only significant for electrons. \\

  \begin{figure}[h!]
    \centering
    % \includegraphics[height = 4cm]{2_LHC_CMS/img/cms_ecal_crystal.png}
    % \includegraphics[height = 4cm]{2_LHC_CMS/img/cms_ecal_endcap.png}
    \caption{Picture of a PbWO$ _4 $ crystal (left) used in the ECAL with its photomultiplier, and of the endcap ECAL (right) showing the crates in which the crystals are placed \Cite{CMS_at_LHC}.}
    \label{fig:lhc_and_cms__cms_ecal_view}
  \end{figure}

  In CMS, the ECAL is composed of PbWO$ _4 $ crystals, acting both as interaction media and as scintillators, attached to photomultipliers to amplify the relatively small amount of photons they emit. The crystals measure 2.2 cm x 2.2 cm, which is equivalent to one Molière radius $ R_M $, by 23 cm, which corresponds to several radiation lengths $ X_0 $. Figure \ref{fig:lhc_and_cms__cms_ecal_view} shows one of these crystals (left), the crates that hold them, and their disposition in the endcap (right). The number of photons collected is proportional to the energy deposited in the calorimeter modulo a correction factor due to the aging of the material. The ambient radiation causes the crystals to become opaque and release less photons which in turn implies a constant need for recalibration of the detectors.

  \section{The Hadronic Calorimeter}

  Where the ECAL relies on radiative processes to detect particles, the HCAL uses strong interactions between the hadrons and the material to create hadronic cascades. These are much longer than electromagnetic showers, requiring longer detectors. The most created particles are pions as they are the lightest hadrons. This induces an electromagnetic component as the $ \pi^0 $ principal decay channel is $ \pi^0 \rightarrow \gamma \gamma $. This creates a problem, as the response of the material can be different for the hadronic and electromagnetic component. \\

  Figures \ref{fig:lhc_and_cms__cms_hcal_view} are a picture of a section of the barrel HCAL representing the absorber (golden plates) with the scintillator in between and of the installation of the barrel HCAL in CMS. The HCAL is composed of an alternation of 16 layers of absorbers, made out of 40 to 70 mm thick steel plates and 50 to 56 mm thick 70\% Cu and 30\% Zn alloy plates, and 3.7 to 9 mm thick plastic scintillators. When particles hit the detectors perpendicularly, they have to travel through 79 cm of matter equivalent to 5.82 interaction lengths $ \lambda_I $. The barrel HCAL is divided into 72 segments in $ \phi $ and 16 $ \eta $ sectors while the endcap HCAL has 36 and 72 $ \phi $ segments for the inners and outers rings respectively, and 14 $ \eta $ sectors.

  \begin{figure}[h!]
    \centering
    % \includegraphics[height = 5cm]{2_LHC_CMS/img/cms_hcal.png}
    % \includegraphics[height = 5cm]{2_LHC_CMS/img/cms_hcal_install.jpg}
    \caption{Picture of the barrel HCAL composed of several dense absorber (golden plates) and smaller scintillators placed in between (left) \Cite{CMS_at_LHC}, and installation in CMS of the barrel HCAL (right) \Cite{CMS_HCAL_Install}}
    \label{fig:lhc_and_cms__cms_hcal_view}
  \end{figure}

  The CMS ECAL's energy resolution is \Cite{CMS_Performances}
  \begin{equation}
    \left( \frac{\sigma_E}{E} \right)^2 = \left( \frac{2.8\%}{\sqrt{E}} \right)^2 + \left( 0.30\% \right)^2 + \left( \frac{0.12}{E} \right)^2 \ ,
  \end{equation}
  where $ E $ is given in GeV. The CMS HCAL's energy resolution is
  \begin{equation}
    \left( \frac{\sigma_E}{E} \right)^2 = \left( \frac{120\%}{\sqrt{E}} \right)^2 + \left( 6.9\% \right)^2 \ .
  \end{equation}


  \section{The Superconducting Magnet}

    \begin{figure}[h!]
      \centering
      \includegraphics[width=\textwidth]{img/I-3-cms/magnet.png}
      \caption{??? \cite{Chatrchyan:2009si}.}
      \label{fig:I-3-cms-magnet}
    \end{figure}

  \section{The Muon System}


  			The intense magnetic field of CMS is created by cooling a solenoid down to 4.5 K, temperature at which the metal becomes supra-conductive, and by passing strong currents through it. The resulting field is uniform inside the solenoid but more complex outside, as shown in Figure \ref{fig:lhc_and_cms__cms_magnetic_field} which represents the measured magnetic field. The constant and strong field in which the tracker is placed allows it to measure the particles' transverse momentum with high-precision (resolution of less than 1\% in the tracker as seen in Figure \ref{fig:lhc_and_cms__cms_tracker_performances}). The intensity of the field is of 3.8 T inside the solenoid and typically 2 T outside the solenoid. \\

  			\begin{figure}[h!]
  				\centering
  				% \includegraphics[width = 12cm]{2_LHC_CMS/img/cms_magnetic_field.png}
  				\caption{Field map of the magnetic field of CMS measured using cosmic rays \Cite{CMS_B_Field}.}
  				\label{fig:lhc_and_cms__cms_magnetic_field}
  			\end{figure}

  			Having the calorimeters inside the magnet improves the energy resolution as particles have less matter to travel through before reaching them, but increases the size of the solenoid. Due to the technical difficulty to build large magnets, the muon chambers are placed on the outside. This layout has the advantage to use the magnet as barrier for most particles escaping the calorimeters, ensuring that only muons will be detected by the muon system.

    \begin{figure}[h!]
      \centering
      \includegraphics[width=\textwidth]{img/I-3-cms/quadrant-postls1.pdf}
      \caption{??? \cite{1748-0221-3-08-S08004}.}
      \label{fig:I-3-cms-quadrant}
    \end{figure}

    \begin{figure}[h!]
      \centering
      \includegraphics[width=\textwidth]{img/I-3-cms/muon-numbering.png}
      \caption{??? \cite{Chatrchyan:2009si}.}
      \label{fig:I-3-cms-muon-numbering}
    \end{figure}


    		Currently, the CMS muon system \Cite{CMS_at_LHC, CMS_Performances} is composed of three different types of gaseous detectors: \emph{Drift Tube} (DT), \emph{Cathode Strip Chamber} (CSC), and \emph{Resistive Plate Chamber} (RPC).

    		\subsection{Disposition of the Detectors}
    		\label{sec:muon_chambers__disposition_of_the_detectors}

    			Like all the CMS detectors, the muon system is divided into two regions: the barrel ($ | \eta | $ < 1) and the endcaps (1 < $ | \eta | $ < 2.4). The chambers are regrouped into stations attached to the wheels of CMS. The barrel stations contain DTs (identified by MBn) and RPCs while the endcaps stations hold CSCs (identified by MEx/y) and RPCs (identified by REn), as represented in Figure \ref{fig:muon_chambers__placement}. For financial reasons, the RPCs were not installed for the LHC's start-up in the 1.6 < $ | \eta | $ < 2.4 region where only CSCs are present.

    			\begin{figure}[h!]
    				\centering
    				% \includegraphics[width = 13cm]{3_Muon_Chambers/img/cms_chambers_placement.png}
    				\caption{Disposition of the muon chambers inside CMS. MBn refer to DTs, MEn to CSCs and the green lines to RPCs \Cite{CMS_Upgrades}.}
    				\label{fig:muon_chambers__placement}
    			\end{figure}

    			The barrel is composed of 5 wheels on which 4 layers of detectors are attached, each divided into 12 stations along $ \phi $. The endcaps have 4 layers of detectors divided into 1, 2 or 3 rings partitioned into 36 or 72 stations that overlap to ensure maximum efficiency. Figure \ref{fig:muon_chambers__cms_endcap} shows the first station of the muon endcap, ME1. The inner ring, called ME1/1 is hidden by the so-called \emph{nose}, in black. The two outer rings, ME1/2 and ME1/3 are well visible. In ME1/2, we can observe the overlap between the chambers. \\

    			\begin{figure}[p!]
    				\centering
    				% \includegraphics[width = 16.5cm]{3_Muon_Chambers/img/cms_endcap_view.jpg}
    				\caption{Picture of one of the endcaps' yokes. We can observe the two outer rings, ME1/2 and ME1/3. Chambers of the inner ring, ME1/1, are hidden inside the \emph{nose}, in black \Cite{Fig_CMS_Endcap}.}
    				\label{fig:muon_chambers__cms_endcap}
    			\end{figure}

    			The use of two different kinds of detectors in each station ensures that the system meets the required detection efficiency for muons imposed by CMS. This redundancy is crucial to select and reconstruct events with high momentum muons in the final state, signature of the Brout-Englert-Higgs boson's decay and of many processes of new physics, including super-symmetry.

    		\subsection{Drift Tubes}
    		\label{sec:muon_chambers__drift_tubes}

    			DTs are rectangular parallelepiped detectors composed of an anode wire stretched between two cathode strips as represented in Figure \ref{fig:muon_chambers__dt}. The chambers are 2.4 m long by 13 mm height by 42 mm wide. A strong electric field (of the order of 1.5 kV cm$ ^{-1} $) is formed by applying a high voltage difference between the electrodes, causing the electrons and ions to drift into the gas, and provoking avalanches near the anode. The two electrodes placed near the anode help flatten the electric field and improve the charges' drift. \\

    			\begin{figure}[h!]
    				\centering
    				% \includegraphics[width = 8cm]{3_Muon_Chambers/img/cms_dt.png}
    				\caption{Schematic view of a drift cell along with the electric field line \Cite{CMS_at_LHC}.}
    				\label{fig:muon_chambers__dt}
    			\end{figure}

    			Four DTs are assembled to create a \emph{Super Layer} (SL), and two or three SLs compose a DT module. Each SL has a spatial resolution of 100 \um{} in the direction perpendicular to the wire. To improve global precision, two SLs are used to measure the $ \phi $ coordinate and sometimes one additional SL is used to measure $ \eta $. DT modules have a time resolution of 3 ns. Their rather large size limits their rate capabilities, explaining why they are only present in the barrel where particles' fluxes are lower (< 10 Hz cm$ ^{-2} $).

    		\subsection{Cathode Strip Chambers}
    		\label{sec:muon_chambers__cathode_strip_chambers}

    			CSCs are trapezoidal multiwire proportional chambers placed in the endcaps of CMS. Multiple anode wires (about 1000 spaced by 3.2 mm) are stretched radially in the chamber above perpendicularly placed cathode strips (typically 80 separated by a pitch of 8.4 mm on the narrow side and 16 mm on the large side) as depicted in Figure \ref{fig:muon_chambers__csc}. As for the DTs, an electric field is formed between the wires and the strips, accelerating the electrons and forming the avalanches near the anodes. By reading-out both electrodes, the CSCs provide a measurement of both coordinates.

    			\begin{figure}[h!]
    				\centering
    				% \includegraphics[height = 4.5cm]{3_Muon_Chambers/img/cms_csc.png}
    				\caption{A representation of a CSC with its wires and strips \Cite{CMS_Performances}}
    				\label{fig:muon_chambers__csc}
    			\end{figure}

    			One CSC module is made out of six chambers put together (7 cathode planes and 6 wire planes). Due to the large number of readout channels in these modules, the spatial resolution is as good as 33 \um{} for ME1/1 and ME1/2, and 80 \um{} for the other stations. The time resolution for one cathode plane is 11 ns that can be brought down to the order of 5 ns when combining the measurements of all the planes. The largest CSC modules, ME2/2 and ME3/2, are 3.4 m by 1.5 m. \\

    			Note that the two dimensional readout configuration can create ambiguities called \emph{ghosts particles} as shown in Figure \ref{fig:muon_chambers__ghosts}. When two particles hit the chamber (left), four possibilities arise when reading the output signal (right). Two of them correspond to real particles, and the two others to ghost particles. It is easy to see that for $ n $ particles interacting in the chamber, $ n^2 $ particles can be reconstructed. This limits the rates at which the CSCs can function to 1 kHz cm$ ^{-2} $.

    			\begin{figure}[h!]
    				\centering
    				% \includegraphics[width = 8cm]{3_Muon_Chambers/img/cms_csc_ghost.png}
    				\caption{Ambiguities arise when more than one particle hit the chamber at the same time.}
    				\label{fig:muon_chambers__ghosts}
    			\end{figure}

    		\subsection{Resistive Plate Chambers}
    		\label{sec:muon_chambers__resistive_plate_chambers}

    			RPCs, represented in Figure \ref{fig:muon_chambers__rpc}, are gaseous parallel plate detectors. They consist of two parallel plates, made out of bakelite with a high resistivity (10$ ^{10} $ to 10$ ^{11} $ $ \Omega $ cm) separated by a gas gap of a few millimeters. The outer surfaces of the resistive materials are coated with conductive graphite to form the HV and ground electrodes. Due to the fact that ions and electrons never come in contact with the electrodes, the evacuation time can be of importance if too many charges are produced. Therefore, the gain of the detectors are reduced and most of the amplification is done by the readout electronics and not by avalanches. This allows the RPCs to run at rates up to 1 kHz cm$ ^{-2} $, while maintaining an excellent time resolution down to 1 ns. On the other hand, they have a poor spatial resolution of the order of 1 mm. \\

    			\begin{figure}[h!]
    				\centering
    				% \includegraphics[width = 10cm]{3_Muon_Chambers/img/cms_rpc.png}
    				\caption{Representation of an RPC with two gas gaps for one readout strip plane \Cite{These_Karol}.}
    				\label{fig:muon_chambers__rpc}
    			\end{figure}

    			Since the RPCs can operate at high hit rate, they are used in both the barrel and the endcaps as trigger system. In the barrel, RPCs are rectangular chambers covering the DTs, while in the endcaps, they have a trapezoidal shape like the CSCs.

    		\subsection{System Performances}
    		\label{sec:muon_chambers__system_performances}

    			Figure \ref{fig:muon_chambers__performances} represents the resolution on the transverse momentum $ p_T $ of muons as a function of the pseudo-rapidity $ \eta $ for the muon system in standalone (left) and combined with the tracker's data (right). The standalone system suffers from discontinuities in $ \eta $ when transitioning from the barrel to the endcaps ($ | \eta | $ = 1) or between stations in the endcaps where particles are not detected. These imprecisions can be removed by considering data from the trackers, which also improves the overall precision by a factor of 10. For a muon with a transverse momentum of 10 \GeVc{}, the resolution goes down from about 10\% in the standalone reconstruction to about 1\% when considering the tracker.

    			\begin{figure}[h!]
    				\centering
    				% \includegraphics[width = 13cm]{3_Muon_Chambers/img/cms_muon_performances.png}
    				\caption{Resolution on the transverse momentum $ p_T $ of muons as a function of the pseudo-rapidity $ \eta $ for the muon system in standalone (left) and combined with the tracker's data (right) \Cite{CMS_Performances}.}
    				\label{fig:muon_chambers__performances}
    			\end{figure}

    \section{The Trigger System}




    		With the LHC running at a rate of 40,000,000 collisions per second, the amount of data produced by CMS is considerable ($ \sim $ 40 TB s$ ^{-1} $). We do not yet have the technology to transfer nor handle all this information. Therefore, a selection of interesting events has to be done in order to reduce the transfer's rate. The decision to keep or drop an event is taken by the CMS trigger system which includes two stages: the \emph{Level-1 Trigger} (L1 Trigger) and the \emph{High Level Trigger} (HLT) \Cite{CMS_at_LHC}. \\

    		Figure \ref{fig:trigger_system_and_reconstruction_algorithms__rates} depicts the different stages of the trigger and the maximal rate of events kept by each one of them, starting at 40 MHz and ending at 100 Hz. The kept events are sent and stored in multiple locations around the world (called \emph{Tiers}) where physicists can analyze them.

    		\begin{figure}[h!]
    			\centering
    			% \includegraphics[width = 10cm]{5_Trigger/img/trigger_rates.png}
    			\caption{Data rates at each stage between the detectors and the data storage center. \Cite{CMS_Trigger_System}}
    			\label{fig:trigger_system_and_reconstruction_algorithms__rates}
    		\end{figure}

    		\subsection{Level-1 Trigger}
    		\label{sec:trigger_system_and_reconstruction_algorithms__level_1_trigger}

    			The L1 Trigger is the first stage of selection of CMS and has to be able to handle all events successively. Therefore, it has to take a "keep or drop" decision every 25 ns (time between two BXs). The system is composed of dedicated electronic chips for each detector placed either on CMS or in the service caverns next to it, to protect them from radiations. A diagram of the L1 Trigger's decision flow is shown in Figure \ref{fig:trigger_system_and_reconstruction_algorithms__l1}. The decision is first taken locally by small groups of muon chambers and calorimeters before being sent to the \emph{Global Muon Trigger} (GMT) \Cite{Trigger_Muon} and \emph{Global Calorimeter Trigger} (GCT) respectively, that analyze the event over all the regions. Finally, the GMT and GCT send their keep/drop signal to the \emph{Global Trigger} (GT). As represented, the tracker is not involved in this first stage selection due to the time needed to reconstruct and transfer the output data. \\

    			\begin{figure}[h!]
    				\centering
    				% \includegraphics[width = 10cm]{5_Trigger/img/l1.png}
    				\caption{L1 Trigger decision flow of CMS before data is being transfered to the DAQ \Cite{CMS_at_LHC}.}
    				\label{fig:trigger_system_and_reconstruction_algorithms__l1}
    			\end{figure}

    			When a collision occurs, every 25 ns, the system reads out the response of every detector and stores it in a buffer that holds the last 128 events. Consequently, the algorithms have a maximum of 3.2 \us{} to process each event and return their decision. This also allows the decision taking process to be differed between the different triggers (GMT and GCT) as the particles take a certain time to travel from the IP to the various detectors. Once an algorithm has made a decision, it sends a one bit signal to the GMT or GCT. When all the detectors have responded, the GT either drops the event or tells the DAQ system to transfer it to the HLT. \\

    			% The fact that the trigger has to take a decision every 25 ns while the data is available for 3.2 \us{} motivates the use of multiple algorithms running in parallel. It is for example possible to execute 128 programs at the same time, each running for 3.2 \us{}. Every 25 ns, a different program would finish its analysis and return a keep or drop signal, allowing the overall system to respond to the L1 Trigger constraints. \\

    			As the L1 Trigger only relies on the calorimeters and on the muon system, the events' selection is done according to the signature and transverse energy left in the calorimeters, and to the transverse momentum reconstructed by the muon system. Only the transverse components of the energy and the momentum are considered as they reflect the physics of the event. Indeed, when the protons collide, the fraction of energy put at play in the interaction is not the same. Therefore, the produced particles will be boosted along \axis{Z} according to these differences, while the total transverse momentum should remain null as the collisions are head to head.

    		\subsection{High Level Trigger}
    		\label{sec:trigger_system_and_reconstruction_algorithms__high_level_trigger}

    			The HLT is composed of a farm of computers running reconstruction software that can perform complex calculations. Due to the filtering made by the L1 Trigger, the incoming data rate is lower (100 kHz), allowing for a longer processing time, of the order of 1 s. If an event passes through the multiple filters and is accepted, it is send to the storage unit and made available for analysis. Since this work aims to study track reconstruction performed at L1, the HLT will not be further reviewed.

    		\subsection{Muon System L1 Trigger}
    		\label{sec:trigger_system_and_reconstruction_algorithms__muon_system_l1_trigger}

    			The different muon chambers have their own trigger system which benefits from the detectors strengths. DTs and CSCs have excellent spatial resolution (respectively of the order of 100 and 80 \um{}) and will therefore be used to filter the events according to their transverse momentum. RPCs on the other hand have great timing capabilities (down to 1 ns) which yields good BX assignment. By combining the DTs and RPCs in the barrel, and the CSCs and RPCs in the endcaps, the GMT can reconstruct event with a multitude of muons in the final state.

    			\subsubsection{Drift Tubes and Cathode Strip Chambers}
    			\label{sec:trigger_system_and_reconstruction_algorithms__dt_and_csc}

    				DTs and CSCs use the same trigger system, the \emph{Track-Finder} (TF) \Cite{Track_Finder}, which relies on the measurement of the particles' bending angle when they pass through the detectors. The system is divided into a local and a global trigger. The local trigger validates hits in DT and CSC modules, before transmitting the information to the global trigger which tries to reconstruct tracks over all the stations.

    				\paragraph*{Local Trigger}
    				\label{sec:trigger_system_and_reconstruction_algorithms__local_trigger}

    					DT modules are divided into smaller segments as represented on the left in Figure \ref{fig:trigger_system_and_reconstruction_algorithms__dt_local}. Each couple of layers (AB, AC, AD, etc) is used to compute the position $ \mathbf{x} $ and the angle $ \phi_b $ of the track by measuring the arrival time of the signals to the anode. The further away a particle passes from the wire, the longer the drift time is, hence the time at which the signals are detected. If the values match for several couples, the segment is considered to represent a valid track and marked as such. The number of couples that return the same value defines the quality of the track. If an ambiguity appears, the parameters with the best quality are selected. \\

    					\begin{figure}[h!]
    						\centering
    						% \includegraphics[height = 3.5cm]{5_Trigger/img/l1_dt_local.png}
    						% \includegraphics[height = 3.5cm]{5_Trigger/img/l1_csc_local.png}
    						\caption{DTs' local trigger measuring the particle's incident angle using the ions' drift time (left) \Cite{CMS_at_LHC}; CSCs' local trigger relying on the multiple layers to measure the particle's incident angle (right) \Cite{Trigger_Muon}.}
    						\label{fig:trigger_system_and_reconstruction_algorithms__dt_local}
    					\end{figure}

    					The same is done in CSC modules using the six planes of anode wires and seven planes of cathode strips, as seen on the right in Figure \ref{fig:trigger_system_and_reconstruction_algorithms__dt_local}. Due to the short amount of time available to run the reconstruction, anode wires are grouped by 5 to 16 by performing a logical \emph{OR} of the binary readout result.

    				\paragraph*{Global Trigger}
    				\label{sec:trigger_system_and_reconstruction_algorithms__global_trigger}

    					Once tracks have been reconstructed locally, the TF matches the different stations by comparing their hits. Figure \ref{fig:trigger_system_and_reconstruction_algorithms__dt_global} shows the pairwise matching between stations (left), the muon track (left; green), and the reconstructed track in the transverse plane (left; red). First, using the local reconstructed angle $ \phi_b $ of the track (middle), the parameters are extrapolated between layers (right) by using predefined parameters stored in LUTs for all the possible matching segments
    					\begin{equation}
    						\phi_{extrapolation} = \phi_b + \phi_{deviation} \ ,
    					\end{equation}
    					where $ \phi_{extrapolation} $ is the extrapolated parameter, and $ \phi_{deviation} $ is the deviation in $ \phi $ between two detection planes. If the extrapolation is close to the measurement, within a predefined range (right; blue), the site is added to the track and the propagation continues towards the next station. This method can result in more than one reconstructed track, which is why only the four tracks with the highest quality (best match between extrapolation and measurements, most matches, etc) are kept. Finally, an estimation of the transverse momentum in function of the bending angle is done.

    					\begin{figure}[h!]
    						\centering
    						% \includegraphics[width = 12cm]{5_Trigger/img/l1_track_finder.png}
    						\caption{Reconstructed trajectory by the Track-Finder using pairwise matching between the stations \Cite{CMS_at_LHC}.}
    						\label{fig:trigger_system_and_reconstruction_algorithms__dt_global}
    					\end{figure}

    			\subsubsection{Resistive Plate Chambers}
    			\label{sec:trigger_system_and_reconstruction_algorithms__rpc_l1_trigger}

    				RPCs use a \emph{Pattern Comparator} (PAC) algorithm \Cite{These_Karol} to retrieve the transverse momentum of the tracks. Each station is divided into segments that are considered to be active if they have been hit, or inactive otherwise. The state of the segments is read out by multiple chips that try to match the hits against a list of preloaded patterns they hold in a LUT. In order to fill the LUT, simulations are ran for a finite number of transverse momenta $ p_T $, associating a code, $ p_T^{Code} $, and a numerous amount of possible patterns to each one of them. For each $ p_T^{Code} $, a ranking is done according to the importance of the patterns
    				\begin{equation}
    					E(pattern, \; p_T^{Code}) = \frac{N_0(pattern, \; p_T^{Code})}{N(p_T^{Code})} \ ,
    				\end{equation}
    				where $ N_0 $ is the number of identical patterns given by the same $ p_T^{Code} $, and $ N $ is the total number of patterns given by the $ p_T^{Code} $. Patterns are added to the LUT, starting with those with the highest $ E $, until 90 to 95\% of the tracks for each $ p_T^{Code} $ are present. \\

    				\begin{figure}[h!]
    					\centering
    					% \includegraphics[width = 6cm]{5_Trigger/img/l1_rpc.png}
    					\caption{Pattern Comparator (PAC) algorithm matching hits against a multitude of predefined patterns \Cite{These_Karol}.}
    					\label{fig:trigger_system_and_reconstruction_algorithms__rpc}
    				\end{figure}

    				Figure \ref{fig:trigger_system_and_reconstruction_algorithms__rpc} shows a couple of patterns that are being tested against measurements. If a layer has not been hit, all the segments are set to active, but the quality of the resulting track will be degraded. The four best reconstructed tracks in the barrel and the endcaps are kept and sent to the GMT.

    			\subsubsection{Global Muon Trigger}
    			\label{sec:trigger_system_and_reconstruction_algorithms__global_muon_trigger}

    				For each event, the GMT receives the four best matches from the TF and from the PAC for the barrel and the endcaps. Those are compared and the best candidates are combined to increase the precision.

    			\subsubsection{System Performances}
    			\label{sec:trigger_system_and_reconstruction_algorithms__system_performances}

    				An important parameter of the L1 Trigger is the applied threshold or cut on the transverse momentum above which all events are accepted. Figure \ref{fig:trigger_system_and_reconstruction_algorithms__acceptance} shows the generated (produced inside CMS; black line) and accepted (reconstructed and accepted by the L1 Trigger; red dots) rate of events according to the applied cut on the transverse momentum $ p_{T; threshold} $ for events with a single muon in the final state. Ideally, the dotted curves should match the continuous line, meaning that the system is able to perfectly identify the muons. Unfortunately, numerous low energy muons are reconstructed with a much higher energy, misleading the trigger. Improving those results would lead to a lower cut as the rates at high energies would drop. The current threshold is at 14 \GeVc{} (blue arrow) which yields a rate of the order of 2 kHz. The selection on a single muon is only one among several tens of L1 Trigger filters. The total bandwidth of the L1 Trigger has to be shared between all the muons, the electrons, the photons, and the jets triggers. Therefore, the maximum event rate allowed to the single muon trigger is limited to 2 kHz. \\

    				\begin{figure}[h!]
    					\centering
    					% \includegraphics[width = 8cm]{5_Trigger/img/l1_acceptance.png}
    					\caption{Generated and accepted rate of events for the L1 Trigger according to their transverse momentum $ p_T $ \Cite{CMS_Performances}.}
    					\label{fig:trigger_system_and_reconstruction_algorithms__acceptance}
    				\end{figure}

    				Figure \ref{fig:trigger_system_and_reconstruction_algorithms__l1_cut} presents the L1 Trigger's efficiency as a function of the transverse momentum $ p_T $ (left) and pseudo-rapidity $ \eta $ (right) for a single muon. The plot on the left is called a \emph{turn-on} plot and shows the acceptance for various $ p_T $ for a defined threshold (14 \GeVc{}). Ideally, the curves should be equal to 0 below the cut and to 1 above the cut. Instead, because of the finite transverse momentum resolution in the muon track reconstruction, the trigger starts to accept events already above 5 \GeVc{}, and only keeps around 95 \% of them above the threshold. The second plot emphasizes what has been reviewed in Section \ref{sec:muon_chambers__ls2_upgrade_and_challenges}, namely the fact that efficiency drops at high $ | \eta | $ due to the lack of redundancy in the muon system.

    				\begin{figure}[h!]
    					\centering
    					% \includegraphics[width = 12cm]{5_Trigger/img/l1_efficiency.png}
    					\caption{L1 Trigger's reconstruction efficiency as a function of the transverse momentum $ p_T $ (left) and pseudo-rapidity $ \eta $ (right) of single muons \Cite{CMS_L1_Efficiency}.}
    					\label{fig:trigger_system_and_reconstruction_algorithms__l1_cut}
    				\end{figure}
