\chapter{The Large Hadron Collider and the CMS Experiment}
\label{chap:lhc_cms}

    The first section of this chapter is dedicated to the \emph{Large Hadron Collider} (LHC), a hadron accelerator and collider. We will present the physics the design of the machine and of the injection chain which provides the LHC with a particle beam, the relevant parameters needed to describe the collisions, and finally the run schedule and upgrades foreseen for the upcoming years. \\

    The second section of this chapter is devoted to the \emph{Compact Muon Solenoid} (CMS) experiment, one of the four main experiments exploiting the LHC beam. We start by giving an overview of the detector and its geometry, and then proceed by describing the sub-detectors that comprise CMS as well as its trigger and \emph{Data Acquisition} (DAQ) system.

    \section{The Large Hadron Collider}

        The LHC \Cite{Evans:1129806} is state-of-the-art in particle accelerators and colliders engineering. Built at and by the \emph{European Organization for Nuclear Research} (CERN), it is located in the 26.7 km long tunnel that previously hosted the \emph{Large Electron Positron} collider (LEP). Using smaller accelerators as injectors, it is designed to accelerate and collide protons or heavy ions at energies up to 14 TeV in the center of mass reference frame. These collisions take place at four different locations where the ALICE \Cite{1748-0221-3-08-S08002}, ATLAS \Cite{1748-0221-3-08-S08003}, CMS \Cite{1748-0221-3-08-S08004}, and LHCb \Cite{1748-0221-3-08-S08005} experiments collect and analyse data.

        \subsection{Machine Design and Injection Chain}

            Explain the injection chain, maybe a bit about the magnets, ...

        \subsection{Beam Structure and Luminosity}

            Explain the beam parameters (beam length, size, density, ...) and the luminosity (relation to cross-section)

        \subsection{Future Plans and Upgrades}

            Explain future upgrade plans for the energy and the luminosity (Phase1, Phase2, ...) and the Long-Shutdowns

    \section{The CMS Experiment}

        Quick info about CMS

        \subsection{Overview and Geometry}

            Objective of CMS (general physics, strong muon, ...)

        \subsection{Inner Tracking System}

            Describe tracking system of post LS1

            \subsubsection{Pixel Detectors}

                Pixels modules

            \subsubsection{Strip Detectors}

                Strips modules

        \subsection{Calorimeters}

            Combine ECAL and HCAL, not so interesting for us

            \subsubsection{Electromagnetic Calorimeter}

                ECAL Blabla

            \subsubsection{Hadronic Calorimeter}

                HCAL Blabla

        \subsection{Superconducting Magnet}

            Quick word about the magnet and the magnetic field

        \subsection{Muon System}

            Design considerations of muon system (far so need to cover big area, . Quick word about gas detectors in general.

            \subsubsection{Drift Tubes}

                DT Duh

            \subsubsection{Cathode Strip Chambers}

                CSC Duh

            \subsubsection{Resistive Plate Chambers}

                RPC Duh

        \subsection{Trigger System}

            Why we need a trigger system

            \subsubsection{Level-1 Trigger}

                About L1

            \subsubsection{High Level Trigger}

                About HLT

            \subsubsection{Muon System Trigger}

                Explain standalone muon trigger

                \paragraph{Local Muon Trigger}

                    LMT Duh

                \paragraph{Global Muon Trigger}

                    GMT Duh

        \subsection{Data Acquisition System}

            Explain principles of DAQ

            \subsubsection{Detectors Readout}

                DAQ readout structure (front-end, back-end in cavern, ...)

            \subsubsection{Event Builder}

                Event reconstruction from multiple subdetectors

            \subsubsection{Event Filter}

                Event filtering (HLT) and storage

            \subsubsection{Run, Control, and Monitoring Infrastructure}

                System control (RCMS, XDAQ, ...)

