\addcontentsline{toc}{chapter}{Abstract}

\begin{abstract}

	In the upcoming years, with the upgrade of the LHC to higher luminosities, CMS will be subject to an increasing flux of particles, especially in its most forward region, at $ | \eta | $ > 1.6. To consolidate the muon spectrometer of CMS and increase redundancy, the CMS GEM collaboration proposes to instrument the 1.6 < $ | \eta | $ < 2.1 region, originally foreseen to be equipped with RPC, with Triple-GEM detectors. This technology has proven to remain efficient under high fluxes and meets the requirements of CMS. \\

	This work, using simulations, studies three track reconstruction algorithms intended to be installed in the first selection stage of CMS for Triple-GEM detectors: a Least Squares fit, a standard Kalman filter, and a modified Kalman filter. A comparative analysis is performed between the results obtained with Triple-GEM detectors and those yielded by the actual system in order to quantify the improvements made to the CMS muon spectrometer and to the CMS trigger system. \\

\end{abstract}

\renewcommand{\abstractname}{Résumé}

\begin{abstract}

	Dans les années à venir, avec la mise à niveau du LHC pour de plus hautes luminosités, CMS sera soumis à un flux croissant de particules, surtout dans la région avant, à $ | \eta | $ > 1.6. Afin de consolider le spectromètre à muons de CMS et d'augmenter la redondance, la collaboration CMS GEM propose d'installer des détecteurs Triple-GEM dans la région 1.6 < $ | \eta | $ < 2.1, qui devait initialement être équipée de RPC. Les Triples-GEMs peuvent resister à des flux intenses sans perdre de leur efficacité et ainsi satisfaire aux exigences de CMS. \\

	Ce travaille étudie, à l'aide de simulations, trois algorithmes de reconstruction de traces destinés à être installés dans le premier niveau de déclenchement de CMS: un Moindres Carrés, un filtre de Kalman standard et un filtre de Kalman modifié. Une analyse comparative est faite entre les résultats obtenus en utilisant les Triple-GEMs à ceux résultant du système actuel afin de caractériser l'impact des différents algorithmes sur le spectromètre à muons et sur le système de déclenchement de CMS.

\end{abstract}

\vfill
\textbf{Keywords:} GEM, Leve1 Trigger, CMS, Upgrade, muon detectors
