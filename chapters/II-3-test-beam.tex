\chapter{A Data Acquisition for the Test Beam}
\label{chap:II-3-test-beam}

  During the ongoing process of the development of the DAQ system, the electronics has been tested in two test beams organised in Fall 2014 and Fall 2015. The first test beam, ran with the first prototype of the DAQ, aimed at proving the feasability of the architecture of the system, not focusing on data taking from the provided pion and muon beam but rather on usability. As the results were encouraging, the second version of the electronics was developped involving a complete redesign of the hardware, firmware, and software. Therefore, we will mainly focus on the second test beam, which electronics is described in the previous chapter, during which abundant data was recorded showing both excellent results for the detectors and the DAQ system. \\

  In this chapter, we present the firmware and software developments done for the DAQ system for the test beams followed by the analysis of the recorded data. First, we will present the firmware architecture of the OH and the GLIB to better understand the global layout of the system and the features that have been implemented to control the components. Then we will move on to the back-end applications developped to control and monitor the DAQ system and read out data. Finally, after a presentation of the layout of the test beam setup and its characteristics, we will show the results obtained after analysis of the recorded data.

  \section{Architecture of the OptoHybrid Firmware}

    The most important function of the OptoHybrid is to transfer data between the VFAT2s and the GLIB. Downwards, from the off-detector to the on-detector electronics, it transfers slow control requests and fast commands while upwards it sends the trigger and tracking data back to the back-end system. Next to the handling of the basic VFAT2 functionnalities, it must also handle the optical links and a couple of programmable registers which control the system. Furthermore, procedures that were previously implemented in software and required extensive computanionnal time have been moved to firmware in order to speed up the system. \\

    Although rather complex, the firmware of the OptoHybrid can be decomposed in five blocks: the slow control, the fast commands, the trigger data, the tracking data, and the optical links. Eventough we will present these blocks separatly, they are tighlty interconnected using a wishbone-like architecture for intercommunication.

    \subsection{Internal Communication Through a Wishbone-Like Architecture}

      Wishbone is an open-source communication protocol used in many projects to enable data transfer between ICs. A light version of the protocol has been implemented in the OptoHybrid to link the various modules of the system. The latter are divided in two categories: masters which initiate requests, and slaves which provide responses. The link between the two is done through a switching hub which redirects the requests to the slaves by using a 32-bit-long address space mapped to the various modules. By design, multiple masters can interact simultaneously with different slaves allowing for parrallelism in the system. \\

      A request form a master is composed of four signals: a flag signaling the presence of a request, a write-enable bit to indicate the nature of the request (read or write), a 32-bit-long address to which the request will be redirected, and an optionnal 32-bit-long data field in case the request is a write operation. The response from a slave consists of: an ackownledgment signal, a 4-bit-long error status in case the operation failed, and an optionnal 32-bit-long data field holding the response to read requests. A programmable timeout has also been implemented to avoir blocking operation.

  \section{Architecture of the GLIB Firmware}

  \section{The Control and Monitoring Application}

  \section{The Test Beam Setup}

  \section{Analysis of the Collected Data}
