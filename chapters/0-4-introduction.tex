\chapter*{Introduction}
\label{chap:0-4-introduction}

  The Large Hadron Collider (LHC), operated since 2008 by the European Organization for Nuclear Research (CERN), is state-of-the-art in the field of particle accelerators and colliders. Providing collisions at an energy of 14 TeV and a luminosity of 10$^{34}$ cm$^{-2}$ s$^{-1}$, it enables scientists to study particle physics at scales never reached before. The data collected by the experiments recording the collisions of the LHC, among which the Compact Muon Solenoid (CMS), is used to test and refine the current models used to describe our universe. To improve the performance of the LHC and increase the recorded statistics, an upgrade of the machine is foreseen for 2018-2019, during the so-called Long Shutdown 2 (LS2). After this upgrade, the LHC will restart with a luminosity of 2 $\times$ 10$^{34}$ cm$^{-2}$ s$^{-1}$, twice its nominal value. This will in turn impact the detectors as the number of particles created during each collision will increase. \\

  The increase in luminosity of the LHC will greatly affect the forward region of the muon spectrometer of CMS were particle rates can reach several kHz cm$^{-2}$. While most of the muon  spectrometer is equipped with two technologies of detectors, either Drift Tubes (DTs) or Cathode Strip Chambers (CSCs) combined with Resistive Plate Chambers (RPCs), the forward region only equips CSCs. Due to concerns regarding the rate capability of the RPCs, the space foreseen to equip these detectors was left vacant, thus relying solely on CSCs to perform measurements. If left as is, CMS will experience of loss of efficiency in the triggering and reconstruction of tracks after LS2. 
