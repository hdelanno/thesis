\chapter*{Introduction}
\label{chap:0-4-introduction}

  The Large Hadron Collider (LHC), operated since 2008 by the European Organization for Nuclear Research (CERN), is state-of-the-art in the field of particle accelerators and colliders. Providing collisions at an energy of 14 TeV and a luminosity of 10$^{34}$ cm$^{-2}$ s$^{-1}$, it enables scientists to study particle physics at scales never reached before. The data collected by the experiments recording the collisions of the LHC, among which the Compact Muon Solenoid (CMS), is used to test and refine the current models used to describe our universe. To improve the performance of the LHC and increase the recorded statistics, an upgrade of the machine is foreseen for 2018-2019, during the so-called Long Shutdown 2 (LS2). After this upgrade, the LHC will restart with a luminosity of 2 $\times$ 10$^{34}$ cm$^{-2}$ s$^{-1}$, twice its nominal value. This will in turn impact the detectors as the number of particles created during each collision will increase. \\

  The increase in luminosity of the LHC will greatly affect the forward region of the muon spectrometer of CMS were particle rates can reach several kHz cm$^{-2}$. While most of the muon  spectrometer is equipped with two technologies of detectors, either Drift Tubes (DTs) or Cathode Strip Chambers (CSCs) combined with Resistive Plate Chambers (RPCs), the forward region only equips CSCs. Due to concerns regarding the rate capability of RPCs, the space foreseen to equip these detectors was left vacant, thus relying solely on CSCs to perform measurements. If left as is, CMS will experience of loss of efficiency in the triggering and reconstruction of tracks after LS2. \\

  To tackle this issue, new detectors relying on the Gas Electron Multiplier (GEM) technology are under development. Studies led by the CMS GEM collaboration have proven that GEM detectors maintain an efficiency above 98\% at particle fluxes as high as 100 MHz cm$^{-2}$. Additionally, they provide a spatial resolution of the order of 150 $\mu$m and a time resolution better than 4 ns with a gas mixture of ArCO$_2$CF$_4$. Through simulations, it was shown that the instrumentation of a layer of GEM detectors in the forward region of the muon spectrometer, coupled with the CSCs, would improve the triggering and reconstruction efficiency of CMS. These results led to the approval of the installation of a full ring of detectors in both endcaps of CMS during LS2. In preparation of the LS2 installation, a small scale test with the near final electronics, called the slice test, will take place starting in March 2017. \\

  Within the GEM collaboration, the Data Acquisition (DAQ) subgroup is in charge of the development of the electronics and software of the DAQ system. The readout and digitization of the GEM detectors is performed by the VFAT3 analog front-end chip coupled to the OptoHybrid board. The latter concentrates the data from the 24 VFAT3s placed on the detector and forwards it over optical links to the off-detector electronics composed of the CTP7. The CTP7 runs within a Micro Telecommunications Computing Architecture (microTCA) crate, standard adopted throughout CMS, and is the gateway to the central DAQ system of CMS. \\

  Besides firmware and software developments, the DAQ system of the GEM detectors has been tested several times during test beam campaigns organized at CERN. Using near final electronics components, the system was used to perform measurements on the detectors and test the reliability of the readout chain. Additionally, characterization and commissioning procedures have been developed to qualify the components for their installation in CMS. These procedures are of importance to understand the response of the analog front-end to the passage of particles in the detector. In parallel, irradiation tests have been performed on the OptoHybrids in order to quantify their sensitivity to upsets created by the high background fluxes to which they will be exposed in CMS. These various tests have helped to build a robust DAQ system that will be used during the slice test and in the system installed during LS2. \\

  The first part of this PhD thesis provides the reader with a background in theoretical and experimental particle physics. Chapter \ref{chap:I-1-standard-model} describes the Standard Model used to describe the fundamental particles and their interactions. Focus is then given, in Chapter \ref{chap:I-2-lhc}, to the technical description and scientific motivations behind the LHC. Chapter \ref{chap:I-3-cms} finalizes the introduction by describing CMS and its subdetectors. \\

  The second part is dedicated to the GEM upgrade project and the thorough description of the its DAQ system and the various tests performed. Chapter \ref{chap:II-1-gem} introduces the reader to the GEM technology and the performance of the detectors. The DAQ system and its evolution are reviewed in Chapter \ref{chap:II-2-daq} in which the electronic components and setups developed over time are detailed. Chapter \ref{chap:II-3-test-beam} presents the firmware and software developments performed by the author for the test beam campaigns organized at CERN in November 2014 and 2015. Additionally, the data recorded during these tests is analyzed and presented, yielding various results on the efficiency of the detectors. Finally, Chapters \ref{chap:II-4-qualification} and \ref{chap:II-5-irradiation} describe the work performed on the qualification and calibration, and the irradiation tests of the on-detector electronics used in the GEM project. \\

  The last part of this work is dedicated to the design of two different architectures of DAQ systems. Chapter \ref{chap:III-1-arch} describe the DAQ system developed for a small prototype of GEM detectors. This system relies on well-known methods used to build DAQ architectures. These are challenged by a novel topology reviewed in Chapter \ref{chap:III-2-web-daq} which changes the interactions whitin a DAQ system. 
