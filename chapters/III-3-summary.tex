\chapter{Summary}
\label{chap:III-3-summary}

  In this part, we presented the work performed on the design of two DAQ systems: one controlling a small Triple-GEM prototype which followed the classical rules of DAQ design, and one that made use of novel technologies to create an innovative architecture. \\

  The two systems implement radically different topologies to interconnect their nodes and handle the flow of data. The readout system of the 10 cm $ \times $ 10 cm GEM prototype, which evolved over time according to the readiness of the hardware components, uses a star network to control the various subsystems. In this configuration, data is pulled from node to node and systematically transits through the central computing node which becomes the bottleneck of the system. The DAQ system relies on a updating scheme with a high frequency of pull requests in order to remain up to date. This in turn increases the traffic on the network and impacts the speed of the communications while diminishing the allocated bandwidth. \\

  The second system implements a two-way mesh network based on novel web technologies which allows for event driven functionalities. Helped by the increase in computational power of microelectronics, the system merges the functions of the central computing node and front-end electronics in a single device which runs a web server. The latter delivers the control and monitoring application to the client and pushes the data in real-time. Furthermore, a direct connection between the server or client and the database can be established to store information. \\

  The architecture created to develop this proof-of-concept has shown to bring improvements on various fronts of data and system management. It provides a truly event-driven architecture which renders obsolete the need of a constant pull mechanism to ensure that all systems are up to date. This in turn reduces the traffic on the network with response times up to ten times faster. The architecture also limits the number of subsystems in the DAQ system and thus maintenance by merging the functionalities of various components together. Finally, relying on a web application to control the system ensures a wider compatibility and lower maintenance cost than system dependent applications.
