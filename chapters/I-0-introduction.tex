In 2013, the Nobel Prize in Physics was awarded to Fran\c{c}ois Englert and Peter W. Higgs ``\textit{for the theoretical discovery of a mechanism that contributes to our understanding of the origin of mass of subatomic particles, and which recently was confirmed through the discovery of the predicted fundamental particle, by the ATLAS and CMS experiments at CERN's Large Hadron Collider}``. For the tens of thousands of physicists who helped made this discovery possible, the news was received as the consecration of a search that lasted 50 years. It started in 1964 when Fran\c{c}ois Englert together with Robert Brout \cite{PhysRevLett.13.321} and, independently, Peter W. Higgs \cite{PhysRevLett.13.508} postulated the existence of a new elementary particle that gives mass to all other particles. They thereby solved the persistant problem of the origin of mass in the theory as it was back then allowing the field of particle physics to leap forward. However, their hypothesis remained unproven until July 4th 2012, day on which two internationnal collaborations of more than 3000 scientist announced that after 2 years of experimentation and data analysis they had discovered the so called \textit{Brout-Englert-Higgs boson} \cite{PhysRevLett.114.191803}, the missing particle which would validate the theory. \\
