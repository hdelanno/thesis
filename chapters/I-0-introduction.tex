Over the last century, with the advent of quantum physics and special relativity, the field of particle physics has made some tremendous progress in the understanding of the elements and the forces that compose our universe. While many questions remain open for future generations to answer, scientists have continuously been improving the models and theories developped to describe the building blocks of nature and their interactions. Benefitting from the technological advancements generated by the apparition of the transistor and modern computing, physicists have been given the tools required to develop more powerful and complex experiments that allow them to confront theoretical predictions to experimental observations while probing deeper the secrets of the universe. \\

In 2013, the Nobel Prize in Physics was awarded to Fran\c{c}ois Englert and Peter W. Higgs ``\textit{for the theoretical discovery of a mechanism that contributes to our understanding of the origin of mass of subatomic particles, and which recently was confirmed through the discovery of the predicted fundamental particle, by the ATLAS and CMS experiments at CERN's Large Hadron Collider}``. For the thousands of physicists who helped make this discovery possible, the news was received as the recognition of a search that lasted over 40 years. It started in 1964 when Fran\c{c}ois Englert together with Robert Brout \cite{PhysRevLett.13.321} and, independently, Peter W. Higgs \cite{PhysRevLett.13.508} postulated the existence of a new elementary particle that gives mass to all other particles. They thereby solved the persistent problem of the origin of mass in the theory as it was formulated, allowing the field of particle physics to leap forward.  However, their hypothesis remained unproven until July 4th 2012, day on which two internationnal collaborations of more than 3000 scientists announced that after 2 years of experimentation and data analysis they had discovered the so called \textit{Brout-Englert-Higgs boson} \cite{PhysRevLett.114.191803}, the missing particle which would validate the theory. \\

The discovery of the Brout-Englert-Higgs boson not only solves the problem of the origin of mass but also strengthens the foundations of the standard model, the prominent theory in particle physics also considered to be one of the greatest achievments in the field. The standard model is a mathematical construction that classifies the elementary particles and the interactions they undergo. It was developped in the 1970s driven by the idea of grand unification, the description of all forces of nature by a single theory, by combining the electroweak and quantum chromodynamics theories, respectivly describing the electromagnetic and weak, and strong interactions. The standard model quickly gained in credibility with the discoveries of neutral weak currents by the Gargamelle experiment and the charm quark by the Stanford Linear Accelerator Center and the Brookhaven National Laboratory in 1974. Over the years, many other theoretical predictions have been proven to be correct through experimental results and discoveries. Those were made possible thanks to the construction of new experiments whose complexity never stops growing and allow physicists to reach new frontiers in their search for answers. \\

This introduction to particle physics aims at providing the reader with a background on both the theoretical and experimental aspects of the field. Chapter \ref{chap:I-1-standard-model} covers the standard model and the mathematical tools used to describe the physics behind it. After a review of the current state of the theory and the description of nature it yields, we address the limitations of the model and list the solutions that are being investigated. We then focus on the experimental side of particle physics in Chapters \ref{chap:I-2-lhc} and \ref{chap:I-3-cms} which respectivly describe the Large Hadron Collider at CERN, the European Organisation for Nuclear Research, and the Compact Muon Solenoid experiment. The Large Hadron Collider is state-of-the-art in particle collider engineering which, through the collisions it provides, enabled the Compact Muon Solenoid detector to discover the Brout-Englert-Higgs boson. We conclude this part in Chapter \ref{chap:I-4-upgrades} by exposing the future of particle physics and the potential discoveries that will be made in the coming years.
