\chapter*{Abstract}
\addcontentsline{toc}{chapter}{Abstract}

  The Gas Electron Multiplier (GEM) upgrade project aims at improving the performance of the muon spectrometer of the Compact Muon Solenoid (CMS) experiment which will suffer from the increase in luminosity of the Large Hadron Collider (LHC). After a long technical stop in 2018-2019, the LHC will restart and run at a luminosity of 2 $\times$ 10$^{34}$ cm$^{-2}$ s$^{-1}$, twice its nominal value. This will in turn increase the rate of particles to which detectors in CMS will be exposed and affect performance. The muon spectrometers in particular will suffer from a degraded detection efficiency due to the lack of redundancy in its most forward region. To solve this issue, the GEM collaboration proposes to instrument the first muon station with Triple-GEM detectors, a technology which has proven to be resistant to high fluxes of particles. Within the GEM collaboration, the Data Acquisition (DAQ) subgroup is in charge of the development of the electronics and software of the DAQ system of the detectors. \\

  This thesis presents the work of the author as lead developer of the firmware for the front-end and back-end electronics. These developments have been performed from the ground up and designed to transfer data from the analog front-end to the off-detector electronics while offering extensive control and monitoring capabilities. The developed DAQ chain has been tested extensively during two test beam campaigns which provided results on both the stability of the system and the efficiency of the detectors. Further characterization of the electronics is described along with the procedures and tools built to qualify the components for their installation in CMS. Finally, the results of the irradiation tests performed on the on-detector electronics are presented. These allowed to understand the effects of radiation on the board and the impact it has on the design of the firmware for CMS. \\

  Additionally, the work of the author on a new architecture for DAQ systems is described. The latter uses modified network topologies and novel web technologies to increase the available bandwidth on the network and yield an event-driven system.
