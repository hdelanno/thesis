\chapter*[Introduction]{Introduction}
\addcontentsline{toc}{chapter}{Introduction}
\label{chap:introduction}

	Before its temporary shutdown early 2013, the \emph{Large Hadron Collider} (LHC) at the \emph{European Organization for Nuclear Research} (CERN) collided protons at energies up to 8 TeV in the center of mass reference frame with an instantaneous luminosity above 10$ ^{33} $ cm$ ^{-2} $ s$ ^{-1} $. Two \emph{Long Shutdowns} (LS) are planned during the first operation phase of the LHC: the first one in 2013-2014 to make the necessary adjustments to reach the nominal energy of 14 TeV in the center of mass reference frame, the second one in 2018-2019 to increase the luminosity beyond the initially foreseen 10$ ^{34} $ cm$ ^{-2} $ s$ ^{-1} $. These maintenance periods offer the possibility to the experiments recording the LHC beam collisions to maintain and upgrade their detectors, and prepare for the high luminosity phase after LS2. \\

	The LHC is designed to collide protons at a frequency of 40 MHz. We do not yet have the technology to handle and store all the produced data at this rate. Inside the \emph{Compact Muon Solenoid} (CMS), one of the four LHC's experiments, each event produces approximatively 1 Megabytes of analyzed data, while the detector generates several Terabytes every second. The maximum amount of data that CMS can store every day is of the order of the Terabytes, yielding a rate of accepted events of 100 Hz, requiring the total rate to be divided by a factor of 4 10$ ^{5} $. \\

	This requires the installation of a trigger system which handles data in real-time, coupled with a complex data acquisition system. The first stage of the CMS's trigger system, called the \emph{Level1 Trigger} (L1 Trigger), analyzes the information from the calorimeters and the muon chambers, using algorithms programmed on dedicated electronics, and performs a first selection of interesting events. It is important to ensure that the system has the capability to recognize the signature of interesting physical processes, while rejecting the other 99.99975\% of the events forever. \\

	The forward region of the CMS muon spectrometer is equipped with two different technologies of gaseous detectors: \emph{Cathode Strip Chambers} (CSC) which yield a good spatial resolution of the order of 100 \um{} and a time resolution of 5 ns, and \emph{Resistive Plate Chambers} (RPC) which offer a lower spatial resolution around 1 mm but an excellent time resolution down to 1 ns. In the most forward region of CMS, the RCPs have not been installed and the L1 Trigger relies on CSCs only. Currently, CMS has the least redundancy, trigger capability, and reconstruction efficiency in the most challenging region for muon detection. High background fluxes and shorter tracks in the transverse plane constitute a challenge when trying to identify muons. \\

	The presence of muons in the final state is a signature of many interesting processes such as the decay of the Brout-Englert-Higgs boson or new physics like super-symmetry. High energy muons often constitue the \emph{golden channel} due to their high detection and reconstruction efficiency. At the higher luminosity at which the LHC will run after LS2, the selection of muons will suffer from an increased background generating coincidences in the detectors and confusing the trigger. With only breadcrumbs of data available, the efficiency of the L1 Trigger will quickly diminish, degrading the performances of the CMS muon spectrometer. \\

	The standard RPCs are not designed to operate at the high rates of particles that will be reached after LS2 and will loose efficiency. New \emph{Gas Electron Multiplier} (GEM) already used in other experiments present the opportunity to equip the vacant region with detectors that have proven to maintain a spatial resolution of the order of 100 \um{}, a time resolution below 5 ns, and a detection efficiency above 98\% even at elevated fluxes. The objective of the CMS GEM collaboration is to instrument the most forward region of the CMS muon spectrometer with Triple-GEM detectors during LS2. \\

	Taking advantage of the improvements made on the performances of dedicated electronics and of the strengths of Triple-GEM detectors, we intend to develop new and more complex algorithms for the L1 Trigger to perform track reconstruction. \\

	The aims of this thesis are the development and the study of three muon track reconstruction algorithms to be run possibly at the L1 Trigger: a Least Squares fit, a standard Kalman filter, and a modified Kalman Filter. To analyze the performance of these algorithms, we have used a self-developed Fast Simulation environment and the official simulation framework of CMS. \\

	Chapter \ref{chap:lhc_and_cms} provides the reader with a general overview of the LHC and CMS, while Chapter \ref{chap:muon_chambers} focuses on the muon spectrometer of the latter. This chapter also reviews the theory behind gas detectors and states the difficulties that will arise once entering the high luminosity phase of the LHC. The GEM technology proposed to be installed as an upgrade in the most forward region of CMS to address these issues is detailed in Chapter \ref{chap:gas_electron_multiplier_detectors}. The Trigger system of CMS and the algorithms currently used to reconstruct and select events are described in Chapter \ref{chap:trigger_system_and_reconstruction_algorithms} which moreover presents various track reconstruction algorithms. \\

	Chapter \ref{chap:simulation_environment} introduces the simulation environments we developed and used to test our reconstruction algorithms. The first tested method, a Least Squares fit, is presented in Chapter \ref{chap:least_squares_fit} along with the obtained results and a thorough analysis of various observable effects. Increasing the complexity of the algorithms, Chapter \ref{chap:kalman_filter} describes the standard Kalman filter and the modified version we implemented. Chapter \ref{chap:algorithms_performances_timing_prospects} compares the algorithms and their impact on the trigger system, and gives some inside on the implementation of the algorithms on a programmable electronics.